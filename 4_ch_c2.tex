\section*{Abstract}
\noindent
Here, we talk about different citation styles and look at tabels.

\noindent{\bf Keywords:} Key1, key2, key3.

\begin{multicols}{2}

\begin{table*}[!htb]
\centering
\caption[Summary of Anderson's Iris Data]{\label{tab:c2t1}
A small summary of Edgar Anderson's Iris Data as implemented in R.}
\begin{small}
\begin{tabular}{ r c c c c }
\multirow{2}{*}{Species} &\multicolumn{2}{c}{Sepal}&\multicolumn{2}{c}{Petal}\\
& Avg. Length & Avg. Width & Avg. Length & Avg. Width\\\hline
\textit{I. setosa}&5.01&3.428&1.46&0.246\\
\textit{I. versicolor}&5.94&2.77&4.26&1.33\\
\textit{I. irginica}&6.59&2.974&5.55&2.03\\\hline
\end{tabular}

\end{small}
\end{table*}

\section{Introduction}

So, as far as citations go, I use a modular command system: This is based on the basic \LaTeX - citation commands (\texttt{$\backslash$cite}, \texttt{$\backslash$citep} and \texttt{$\backslash$citealt}) and two single-letter suffixes (\texttt{i}/\texttt{a}/\texttt{b}/\texttt{c}/\texttt{d}/\texttt{z} and \texttt{t}/\texttt{m}). The first letter indicates the current section (Intro: \texttt{i}, Chapter 1:\texttt{a}, 2:\texttt{b}, Synthesis:\texttt{z}), while the last letter indicates whether this is the self-citation from the Chapter title page (\texttt{t}) or a citation within the manuscript (\texttt{m}).
Form these pieces, the actual commands can be puzzled together, with the actual commands looking something like \texttt{$\backslash$citepcm\{\}}.

So, you might remember that in the first Chapter \citepbm{ref1}, we cited \citebm{Darwin1859} a lot.
Now in this chapter we are going to look at the data from \citealtbm{Anderson35}.
We are going to show a summary of this data in a main table (\tabref{tab:c2t1}), while a glimpse of the actual data structure is given in a Supplemental Table (\tabrefS{tab:c2st1}).
The referencing of the tables happens using the commands \texttt{$\backslash$tabref\{\}} and \texttt{$\backslash$tabrefS\{\}}.

\section{Even More Dummy Content}
\textcolor{black!35}{\lipsum[7-8]}

\begin{supplTable*}[!t]
\centering
\captionsetup{width=.9\linewidth}
\caption[Subset of Anderson's Iris Data]{\label{tab:c2st1}
A subset of Edgar Anderson's Iris Data as implemented in R.}
\begin{small}
\begin{tabular}{ r c c c c }
\textbf{Species}&\textbf{Sepal Length}&\textbf{Sepal Width}&\textbf{Petal Length}&\textbf{Petal Width}\\\hline
\textit{I. setosa}&5.4&3.4&1.7&0.2\\
\textit{I. setosa}&5.4&3.7&1.5&0.2\\
\textit{I. setosa}&5.7&3.8&1.7&0.3\\
\textit{I. setosa}&5.1&3.5&1.4&0.2\\
\textit{I. setosa}&4.8&3&1.4&0.3\\
\textit{I. setosa}&5.1&3.3&1.7&0.5\\\arrayrulecolor{black!25}\hline
\textit{I. versicolor}&5.8&2.6&4.0&1.2\\
\textit{I. versicolor}&5.8&2.7&3.9&1.2\\
\textit{I. versicolor}&5.5&2.3&4.0&1.3\\
\textit{I. versicolor}&6.0&2.7&5.1&1.6\\
\textit{I. versicolor}&5.6&3.0&4.5&1.5\\
\textit{I. versicolor}&6.3&3.3&4.7&1.6\\\hline
\textit{I. virginica}&6.9&3.2&5.7&2.3\\
\textit{I. virginica}&7.2&3.2&6.0&1.8\\
\textit{I. virginica}&6.5&3.0&5.5&1.8\\
\textit{I. virginica}&5.8&2.7&5.1&1.9\\
\textit{I. virginica}&6.3&3.4&5.6&2.4\\
\textit{I. virginica}&7.7&2.6&6.9&2.3\\\arrayrulecolor{black}\hline\hline
\end{tabular}

\end{small}
\end{supplTable*}

\section{I Can Not Take the Dummies Anymore}
\textcolor{black!35}{\lipsum[10-11]}

\section*{Acknowledgments}

We thank all the nice people.

\bibliographystylebm{apalike}
\bibliographybm{library/c2.bib}
\end{multicols}
